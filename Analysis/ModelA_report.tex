\documentclass{article}
\usepackage{graphicx}
\usepackage{amsmath}

\begin{document}

\title{Simple PFC Model}
\author{Emily Wilson}

\maketitle


\section*{Model A}
Model A systems are systems where the order parameter is not conserved, a common example is the Ising model of a spin system. Throughout the evolution of the system the total number of "up" spins does not need to stay constant. This system can be modelled by the following equations:
\begin{equation*}
\phi ^ {n + 1} (i,j) = \phi^n(i,j) + \frac{\Delta\bar{t} }{\Delta \bar{x} ^2} \bar{\Delta}^2 \phi^n{n+1} (i, j)- \Delta \bar{t} \frac{\partial f(\phi ^n (i,j))}{\partial{\phi}} \\
\end{equation*}
with $\bar{\Delta}^2$ is the discreet Laplacian which may be given by
\begin{equation*}
\bar{\Delta}^2 \phi ^n (i,j) = \phi(i+1, j) + \phi(i-1, j) + \phi(i, j + 1) + \phi(i, j - 1) - 4 \phi(i,j) 
\end{equation*}
although there are other ways to represent the discreet Laplacian that in some cases may be better suited. Used periodic boundary conditions.  The free energy functional may be expressed as a function with two minima, $f(\phi^n(i,j)) = a_1 + \frac{a_2}{2} (\phi^n(i,j))^2 + \frac{a_4}{4} (\phi^n(i,j))^4 $ which has the following first order derivative
\begin{equation*} 
\frac{\partial{\phi^n(i,j)}}{\partial {\phi}} = a_2 (\phi^n(i,j))^2 + a_4 (\phi^n(i,j))^3
\end{equation*}\\ \\
\textbf{Notes on variables:}\\
$\bullet$ $i, j = 0 .. N$ are indices for the $\phi$ array which represents the phase field of \indent the system. \\
$\bullet$ $N$ is the size of the array, here the system is assumed to have the same size \indent in $x$ and $y$ directions.\\
$\bullet$  $n = 0 .. T$ is the time index of the sytem.\\
$\bullet$  $T$ is the final time of the array.\\
$\bullet$ $\bar{t} = n M \Delta t$, $M$ is the mobility constant which is analogous (similar ?) to the diffusion coefficient\\ 
$\bullet$ $\bar{x} = n \frac{\Delta x}{W_{o}}$, $W_{o}$ is the interface width (?)\\
$\bullet$ $\Delta \bar{t}  <  \frac{1}{4} \Delta \bar{x}$, euler time marching algorithm stability restriction in 2D
\\




\end{document}